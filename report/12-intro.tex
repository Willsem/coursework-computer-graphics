\Introduction

Целью данной работы является разработка метода синтаксического анализа на основе семантических категорий. Для достижения поставленной цели необходимо решить следующие задачи:

\begin{itemize}
\item проанализировать существующие методы синтаксического анализа;
\item проанализировать строение языка;
\item сформулировать понятие семантических категорий;
\item разработать программу для извлечения семантических категорий;
\item проверить её работоспособность.
\end{itemize}

\Abbrev{КЛ}{Компьютерная лингвистика}
\Abbrev{ЕЯ}{Естественный язык}
\Abbrev{РЯ}{Русский язык}
\Abbrev{ЛК}{Лингвистический корпус}
\Define{Флективный язык}{язык синтетического типа, в котором доминирует словоизменение при помощи флексий.}
\Define{Синтетический язык}{типологический класс языков, в которых преобладают синтетические формы выражения грамматических значений.}
\Define{Грамматика зависимостей} {одна из формальных грамматических моделей, представляющая строй предложения в виде иерархии компонентов, между которыми установлено отношение зависимости.}
\Define{N-грамма} {последовательность из n элементов.}
\Define{Лингвистический корпус} { это некоторый филологически-компетентный массив языковых данных (как правило, множество текстов), отобранных в соответствии с некоторой исследовательской задачей и специально подготовленных, размеченных, структурированных, представленных в унифицированном виде.}
\Define{Лексема} {слово, рассматриваемое как единица словарного состава языка в совокупности всех его конкретных грамматических форм и выражающих их флексий, а также всех возможных значений.}
\Define{Лемитизация} {выделение лексем.}
\Define{Токен} {отдельное слово, фраза или любой другой значимый элемент текста.}
\Define{Токенизация} {выделение границ слов, предложений, абзацев и других элементов текста.}