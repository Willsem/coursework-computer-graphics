
%\NirEkz{Экз. 3}                                  % Раскоментировать если не требуется
%\NirGrif{Секретно}                % Наименование грифа

%\gosttitle{Gost7-32}       % Шаблон титульной страницы, по умолчанию будет ГОСТ 7.32-2001, 
% Варианты GostRV15-110 или Gost7-32 
 
\NirOrgLongName{ 
МОСКОВСКИЙ ГОСУДАРСТВЕННЫЙ ТЕХНИЧЕСКИЙ УНИВЕРСИТЕТ ИМ. Н. Э. БАУМАНА
}                                           %% Полное название организации

\NirUdk{УДК № 004.822}
%\NirGosNo{№ госрегистрации }
%\NirInventarNo{Инв. № ??????}

%\NirConfirm{Согласовано}                  % Смена УТВЕРЖДАЮ
\NirBoss[.49]{Проректор университета\\по научной работе}{В.Н. Зимин.}            %% Заказчик, утверждающий НИР


%\NirReportName{Научно-технический отчет}   % Можно поменять тип отчета
%\NirAbout{О составной части \par опытно-конструкторской работы} %Можно изменить о чем отчет

%\NirPartNum{Часть}{1}                      % Часть номер

%\NirBareSubject{}                  % Убирает по теме если раскоментить

% \NirIsAnnotacion{АННОТАЦИОННЫЙ }         %% Раскомментируйте, если это аннотационный отчёт
%\NirStage{промежуточный}{Этап \No 1}{} %%% Этап НИР: {номер этапа}{вид отчёта - промежуточный или заключительный}{название этапа}
%\NirStage{}{}{} %%% Этап НИР: {номер этапа}{вид отчёта - промежуточный или 

\Nir{}

\NirSubject{Генерация объемных кучевых облаков в реальном времени}                                   % Наименование темы
%\NirFinal{}                        % Заключительный, если закоментировать то промежуточный
%\finalname{итоговый}               % Название финального отчета (Заключительный) 
%\NirCode{Шифр\,---\,САПР-РЛС-ФИЗТЕХ-1} % Можно задать шифр как в ГОСТ 15.110
\NirCode{}

% \NirManager{Зам. проректора по научной работе}{Р.А. Бадамшин  } %% Название руководителя
\NirIsp{Руководитель темы}{Антон Александрович Оленев} %% Название руководителя

\NirYear{2019}%% если нужно поменять год отчёта; если закомментировано, ставится текущий год
\NirTown{Москва}                           %% город, в котором написан отчёт
